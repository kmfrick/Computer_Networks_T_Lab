\documentclass{beamer}
\usepackage{amsmath}
\usepackage[italian]{babel}
\usepackage{commath}
\usepackage[autostyle, english = american]{csquotes}
\usepackage{graphicx}
\usepackage{parskip}
\usepackage{subcaption}
\usepackage{minted}
\usetheme{Boadilla}
\begin{document}
\title{Esercitazione 2}
\subtitle{Invio di direttori in Java}
\author{Corradini, De Luca, di Nuzzo, Frick, Ragazzini}
\institute{unibo}
\date{2019}
\setbeamercovered{transparent}
\begin{frame}
\titlepage
\end{frame}
\begin{frame}[fragile]{Introduzione}
\textbf{Obiettivo:} Sviluppo di un'applicazione cliente/servitore che si occupi di copiare i file da uno o più direttori selezionati dall'utente all'interno del file system del server, purché la loro dimensione sia maggiore di una data soglia.

\textbf{Struttura:} Un \texttt{FatherServer} lancia un \texttt{ChildServer}, estensione di \texttt{Thread}, per ogni richiesta di connessione; questi ricevono i metadati e i contenuti dei file dal \texttt{Client} tramite socket Java conformi al protocollo TCP.
\end{frame}
\begin{frame}[fragile]{Client}
Funzione: trasferimento di file al servitore.

\textbf{Protocollo di trasferimento:}

Per ogni file nella directory (\textit{range-based for loop}):
\begin{itemize} 
\item Invio nome (\texttt{writeUTF()}) del file;
\item Ricezione risposta (file già presente?);
\item Invio dimensione (\texttt{writeLong()}) del file;
\item Invio del file.
\end{itemize}
Viene chiamato \texttt{flush()} solo dopo aver scritto l'intero file sullo stream, per poter inviare un file intero.
\end{frame}

\begin{frame}[fragile]{FatherServer}
Funzione: demone in grado di creare processi concorrenti che gestiscano l'elaborazione delle richieste.

Il FatherServer crea i thread e assegna ad ognuno di loro una socket TCP. 
\end{frame}

\begin{frame}[fragile]{ChildServer}
Il \texttt{ChildServer} riceve dalla socket il nome del file e controlla che esso non sia già presente nel proprio file system. Nel caso non lo sia riceve anche la dimensione e i contenuti del file.

Indicazione di fine direttorio: \texttt{readUTF()} lancia \texttt{EOFException} quando sono finiti i file da inviare.
\end{frame}

\begin{frame}{Conclusioni}
Difficoltà riscontrate:
\begin{itemize}
\item Garantire la sincronia tra cliente e servitore;
\item Trasferimento di file non testuali;
\item Indicazione di fine trasferimento direttorio.
\end{itemize}
\end{frame}
\end{document}
