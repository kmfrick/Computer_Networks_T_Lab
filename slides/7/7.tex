\documentclass{beamer}
\usepackage[italian]{babel}
\usepackage[autostyle, english = american]{csquotes}
\usepackage{graphicx}
\usepackage{parskip}
\usepackage{subcaption}
\usetheme{Boadilla}
\graphicspath{ {./} }
\begin{document}
\title{Esercitazione 7}
\subtitle{RMI Registry Remoto come pagine gialle}
\author{Corradini, De Luca, di Nuzzo, Frick, Ragazzini}
\institute{unibo}
\date{2019}
\setbeamercovered{transparent}
\begin{frame}
\titlepage
\end{frame}

\begin{frame}{Registry Remoto}
Il registry remoto fa da servizio di nomi per il cliente: esso ha la possibilità di richiedere tramite opportune funzioni la lista dei nomi dei servizi corrispondenti a un dato tag.
Con il nome del servizio il cliente può ottenere il riferimento remoto.
Il servitore ha la possibilità di potersi registrare e associare il proprio tag.
\end{frame}

\begin{frame}{Cliente}
Il cliente interroga il registry remoto per ottenere tutti i servitori corrispondenti a un tag richiesto da stdin all'utente.
Dopo daver ottenuto l'elenco dei servitore richiede all'utente di sceglierne uno e si connette per la gestione del programma congresso.
\end{frame}

\begin{frame}{Servitore}
I servitori dovranno connettersi e registrarsi tramite la funzione remota fornita col Registry remoto.
Dopo essersi connessi impostano il proprio tag ed entrano in attesa di richiesta di registrazione e stampa da parte del cliente.
\end{frame}

\begin{frame}{Conclusioni}

Si sarebbe potuto utilizzare una "lista di liste" strutturata dividendo la prima lista in tag dove ogni elemento della stessa contiene i riferimenti remoti di tutti i servitori con quel tag.
Tuttavia si è scelto di usare una matrice statica per soddisfare la richiesta di utilizzare costrutti semplici.

\end{frame}

\end{document}
