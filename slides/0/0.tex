\documentclass{beamer}
\usepackage{amsmath}
\usepackage[italian]{babel}
\usepackage{commath}
\usepackage[autostyle, english = american]{csquotes}
\usepackage{parskip}
\usepackage{subcaption}
\usetheme{Boadilla}
\begin{document}
\title{Esercitazione 0}
\subtitle{Lettura e scrittura di file in C e Java}
\author{Corradini, di Nuzzo, Frick, Ragazzini}
\institute{unibo}
\date{2019}
\setbeamercovered{transparent}
\begin{frame}
	\titlepage
\end{frame}
\begin{frame}
\frametitle{Codice C e Java: produttore}
Il produttore è il processo che crea il file e, dopo aver letto \textit{carattere per carattere} il contenuto inserito dall'utente fino alla ricezione di \texttt{EOF}, lo scrive sul file di testo.

\textbf{Logica di soluzione}
\begin{itemize}
\item Controllo input;
\item Apertura/creazione del file in scrittura;
\item Ciclo di lettura da standard input:
\begin{itemize}
\item inserimento in un buffer temporaneo del singolo carattere letto;
\item scrittura su file del contenuto del buffer;
\end{itemize}
\item Chiusura del file.
\end{itemize}
\end{frame}

\begin{frame}
\frametitle{Codice C: consumatore}
Il consumatore è il processo che legge un file carattere per carattere, filtrandone il contenuto in base a una lista di caratteri passati come parametro.

\textbf{Logica di soluzione}
\begin{itemize}
\item Controllo input;
\item Inizializzazione file descriptor:
\begin{itemize}
\item standard input,
\item apertura file;
\end{itemize}
\item Lettura a carattere:
\begin{itemize}
\item se non presente nella stringa di filtro, stampa il carattere;
\end{itemize}
\item Chiusura file.
\end{itemize}
\end{frame}
\begin{frame}
\frametitle{Codice Java: consumatore}
Il consumatore è il processo che legge un file carattere per carattere, filtrandone il contenuto in base a una lista di caratteri passati come parametro.


\textbf{Logica di soluzione}
\begin{itemize}
\item Controllo input;
\item Creazione istanza Reader:
\begin{itemize}
\item standard input (InputStreamReader),
\item apertura file (FileReader);
\end{itemize}
\item Lettura a carattere (da BufferedReader):
\begin{itemize}
\item se non contenuto nella stringa di filtro, stampa il carattere;
\end{itemize}
\item Chiusura Reader.

\end{itemize}
\end{frame}
\end{document}
