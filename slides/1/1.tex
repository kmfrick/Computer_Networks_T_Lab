\documentclass[8pt]{beamer}
\usepackage{amsmath}
\usepackage[italian]{babel}
\usepackage{commath}
\usepackage[autostyle, english = american]{csquotes}
\usepackage{parskip}
\usepackage{subcaption}
\usepackage{minted}
\usetheme{Boadilla}
\begin{document}
\title{Esercitazione 1}
\subtitle{Scambio di righe in Java}
\author{Corradini, di Nuzzo, Frick, Ragazzini}
\institute{unibo}
\date{2019}
\setbeamercovered{transparent}
\begin{frame}
\titlepage
\end{frame}
\begin{frame}[fragile]{RowSwapServer - 1}
\vspace{0.9em}
\begin{itemize}
\linespread{1.2}
\item Il \texttt{RowSwapServer} implementa la logica di soluzione vera e propria.
Vi è un \texttt{RowSwapServer} per ogni file da modificare.
\item Nel costruttore viene aperta la socket con la quale comunicherà con il \texttt{DiscoveryServer} e gli vengono indicati il nome del file su cui deve operare e la porta sulla quale aprire la socket. 
\item Il \texttt{RowSwapServer} estende \texttt{Thread} e implementa la logica di soluzione nel metodo \texttt{run()}.
\vspace{1.5em}
\end{itemize}
\begin{minted}{java}
public RowSwapServer(String filename, int port) throws SocketException, FileNotFoundException {
  socket = new DatagramSocket(port);
  packet = new DatagramPacket(buf, SIZE);
  socket.setSoTimeout(30000);
  filename = filename;
  port = port;
  Random r = new Random(System.currentTimeMillis());
  outfile = filename + ".out" + r.nextInt(256);
  success = 0;
  System.out.println("RSS: Ricevuto " + filename + " " + port);
  File file = new File(filename);
  if (!file.exists() || file.length() == 0)
    throw new FileNotFoundException();
}
\end{minted}
\end{frame}
\begin{frame}[fragile]{RowSwapServer - 2}
\begin{itemize}
\item Per scambiare le righe il file viene prima letto fino alla seconda riga coinvolta, salvando in memoria i contenuti delle due righe (omessa la parte di codice per la gestione delle eccezioni, per leggibilità).
\end{itemize}
\small
\begin{minted}{java}
while (i < b) {
  line = reader.readLine();
  if (line == null) throw new IOException();
  else if (i == a) firstLine = line;
  i++;
}
if (a != b) secondLine = reader.readLine();
else secondLine = firstLine;
\end{minted}
\normalsize
\begin{itemize}
\item Il file viene poi scorso di nuovo e stampato riga per riga, avendo cura di intercettare le righe in esame e scambiarle. 
\end{itemize}
\small
\begin{minted}{java}
while ((line = reader.readLine()) != null) {
  if (i == a) {
    System.out.println(secondLine);
    writer.write(secondLine);
  } else if (i == b) {
    System.out.println(firstLine);
    writer.write(firstLine);
  } else {
    System.out.println(line);
    writer.write(line);
  }
  writer.newLine(); // Portabilità, invece di \n
  i++;
}
\end{minted}
\normalsize
\vspace{0.6em}
Per poter ricominciare la lettura del file dall'inizio viene utilizzato un \texttt{FileInputStream} invece di costruire il \texttt{BufferedReader} leggendo direttamente il file.
\end{frame}
\begin{frame}[fragile]{DiscoveryServer - 1}
\begin{itemize}
\item Il \texttt{DiscoveryServer} lancia i \texttt{RowSwapServer} corrispondenti a ogni file come \texttt{Thread} separati.
\end{itemize}
\begin{minted}{java}
i = 1;
while (i < args.length) {
  port = Integer.parseInt(args[i + 1]);
  RowSwapServer rowSwapServer = new RowSwapServer(args[i], port);
  rowSwapServer.start();
  i += 2;
}
\end{minted}
\begin{itemize}
\item Gli argomenti, nella forma di coppie \texttt{"nomefile porta"} vengono elaborati due alla volta: l'indice del ciclo, di conseguenza, assume solo valori dispari.
\end{itemize}
\begin{minted}{java}
i = 1;
port = -1;
while (i < args.length) {
  if (args[i].equals(nomefile))
    port = Integer.parseInt(args[i + 1]);
  i += 2;
}
\end{minted}
\end{frame}
\begin{frame}[fragile]{DiscoveryServer - 2}
\begin{itemize}
\item In seguito a una richiesta, se essa corrisponde a un \texttt{RowSwapServer} valido, il \texttt{DiscoveryServer} risponde con la porta del server in esame.
\end{itemize}
\begin{minted}{java}
ByteArrayOutputStream boStream = new ByteArrayOutputStream();
DataOutputStream doStream = new DataOutputStream(boStream);

doStream.writeUTF(Integer.toString(port));
clientComunicationBuffer = boStream.toByteArray();
clientPacket.setData(clientComunicationBuffer);
clientSocket.send(clientPacket);

clientSocket.disconnect();
clientSocket.close();

\end{minted}
\end{frame}
\begin{frame}[fragile]{Client}
\begin{itemize}
\item Il \texttt{Client} chiede al \texttt{DiscoveryServer} di essere messo in comunicazione con un \texttt{RowSwapServer}.
\item Al \texttt{RowSwapServer}, invece, chiede di scambiare due righe di un file.
\item Attraverso la socket condivisa con il \texttt{DiscoveryServer} viene inviato il numero della porta da utilizzare per la comunicazione con il \texttt{RowSwapServer}.
\end{itemize}
\vspace{1em}
\begin{minted}{java}
// Chiede la porta a DS
packet.setData(data);
socket.send(packet);
packet.setData(buf);
socket.receive(packet);
ByteArrayInputStream biStream = 
new ByteArrayInputStream(packet.getData(), 0, packet.getLength());
DataInputStream diStream = new DataInputStream(biStream);
int portRS = Integer.parseInt(diStream.readUTF());
// Il discovery server lancia un valore negativo se non trova il file
if (portRS < 0) { 
  System.err.println("File non trovato.");
  System.exit(3);
}
\end{minted}
\end{frame}
\end{document}
